\documentclass[10pt,a4paper]{article}
\usepackage[latin1]{inputenc}
\usepackage[spanish]{babel}
\begin{document}

\title{Metodolog\' ias \' Agiles y Tradicionales}
\author{Jonathan Le\'on \\Francisca Sapiains\\Cristian Mondaca}
\maketitle

La ingenier\' ia de software, es una disciplina que ha estudiado propuestas diferentes
 que permiten apoyar el proceso de desarrollo y construcci\' on de software 
 desde hace muchos a\~nos sin embargo muchos proyectos de desarrollo de
  software no llegan a cumplir su cometido,
  ya sea porque simplemente fracasan o el resultado final obtenido no es el
   solicitado por los clientes, usuarios finales o incluso los mismos 
   desarrolladores, debido, por ejemplo, a que el
   software que debieron crear fue mas complejo de lo que imaginaron al aceptar el proyecto,
    lo cual influye de manera negativa en la calidad del programa entregado, en no poder 
    cumplir con los tiempos estimados con el cliente o incluso, que el software 
    entregado no cumpla las funciones que la empresa solicito, es decir, un fracaso. \\

En la actualidad se han encontrado varios factores que pueden llevar al fracaso
 de un proyecto de desarrollo de software, los cuales los dividimos en dos grandes
  grupos: metodolog\' ias
   tradicionales y metodolog\' ias \' agiles. Una metodolog\' ia tradicional 
   se caracteriza por tener equipos de trabajo grandes, con roles muy definidos 
   por los participantes, rigidez en sus fases de trabajo, escasa comunicacion 
   con el cliente y poseer una excesiva documentaci\' on durante la vida del proyecto.
    En cambio una metodolog\' ia \' agil se caracteriza por tener equipos de desarrollo 
    auto-organizados, trabajo a corto plazo, comunicaci\' on con el cliente, valoraci\' on 
    al cambio, mostrar avances del trabajo al cliente y proyectos
     poco documentados debido a la minimizaci\' on de tareas que no contribuyen al software. \\\\
Metodolog\' ia Tradicional (METODOLOGIA RUP)\\\\ %aprender a poner como subtitulo
RUP (Rational Unified Process) es una metodolog\' ia de desarrollo de software 
tradicional que junto con UML constituye la metodolog\' ia est\' andar 
para el an\' alisis de sistemas orientados a objetos. RUP es guiado
 por los casos de uso generados en 
los modelos UML,  La metodolog\' ia RUP se caracteriza por su forma 
disciplinada de asignar las tareas y responsabilidades por cumplir 
(quien hace que, cuando y como), posee un desarrollo iterativo e incremental, 
modelado visual del software, verificaci\' on de calidad del software, control de 
cambios, administraci\' on de requisitos entre otras caracter\' isticas de la metodolog\' ia. 
El RUP esta basado en 6 principios base:\\\
\begin{enumerate}
\item adaptar el proceso: adaptar seg\' un necesidades del cliente.
\item equilibrar prioridades: los requisitos de los participantes
 pueden ser diferentes o pueden disputar recursos que se encuentran limitados. 
\item Demostrar valor iterativamente: los proyectos se entregan 
en etapas iteradas. 
\item Colaboraci\' on entre equipos: El desarrollo del software
 lo realizan m\' ultiples equipos. 
\item elevar nivel de abstracci\' on: Uso de conceptos reutilizables
 como patr\' on del software, frameworks, entre otros. 
\item Enfocarse en la calidad: el control de calidad debe 
realizarse en todos los aspectos de producci\' on\\\\

\end{enumerate}

 El m\' etodo de desarrollo de RUP se divide en 4 fases:\\\
\begin{enumerate}

\item Fase de inicio: su prop\' osito es definir y 
acordar el alcance del proyecto con los clientes, identificar 
riesgos del proyecto, generar un plan para las fases e iteraciones. \\\
\item Fase de elaboraci\' on: Se dise\~na una soluci\' on preliminar, 
seleccionan y desarrollan los casos de uso y se realiza
 el primer an\' alisis del problema a resolver. \\\
\item Fase de desarrollo: como dice su nombre se 
empieza a desarrollar la elaboraci\' on del software
 para la cumplir con los requisitos que exige el cliente,
  administrar cambios de acuerdo a evaluaciones de usuarios
   y realizar mejoras al proyecto. \\\
\item Fase de Transici\' on: su prop\' osito es asegurar
 la funcionalidad adecuada del software para ser entregado
  al usuario final, ajustar defectos mediante pruebas
   y proveer soporte t\' ecnico.\\\\

\end{enumerate}

Entre las ventajas de usar RUP se encuentra la reducci\' on 
de riesgos del proyecto, el que integra desarrollo con
 mantenimiento e incorpora fielmente el objetivo de calidad.\\\\\\
Metodolog\' ia \' agil  (EXTREME PROGRAMMING, XP)\\\\
Esta metodolog\' ia basa su funcionamiento en resaltar 
las distintas relaciones humanas para poder sacar el m\' aximo
provecho de los individuos detr\' as
 de un proyecto, con el fin de obtener los mejores resultados a la hora de crear
un software.  Algunas de las
 caracter\' isticas de este tipo de metodolog\' ia \' agil es que est\' a
  centrado en el aprendizaje
constante de los desarrolladores,
 adem\' as que est\' en insertos en un ambiente laboral grato, para que se generen 
instancias favorables que hagan 
estimular el trabajo en equipo de los distintos personajes de un proyecto en 
particular. \\
XP esta creado para que el cliente y 
el grupo de trabajo est\' en en constante comunicaci\' on, donde el cliente informe
 los cambios que requiere su software 
 y  se la haga saber al grupo de trabajo apenas nace esta inquietud. El grupo
de trabajo responde con soluciones
 concretas y simples donde particularmente se enfoca el trabajo en entregar al 
usuario lo que desea, es por esto 
que la comunicaci\' on tiene que ser fluida entre las partes. La particularidad de
esta metodolog\' ia es que se puede 
aplicar a proyectos donde los cambios en el dise\~ no o en los requerimientos est\' an
a la orden del d\' ia. \\
Las historias de usuarios son
 una parte fundamental dentro de esta metodolog\'ia, ya que se le denomina historia a
peque\~ nos fichas de papel,
 donde el usuario escribe en forma breve y simple los requisitos funcionales o no 
funcionales que debe tener el
 software. Estas historias
  pueden modificarse o eliminarse en cualquier minuto, est\' an
sujetas a cambios permanentes durante el proyecto. 
Los distintos roles de XP, son 
los siguientes. Programador, Cliente,  Encargado de pruebas, Encargado de seguimiento,
Entrenador, Consultor y gestor.\\
\\
Para finalizar mostraremos el
 proceso de la metodolog\' ia.\\
Proceso XP\\
Consiste principalmente en los siguientes bloques: \\
\begin{enumerate}
\item El cliente define el valor de negocio a implementar.
\item    El programador estima el
 esfuerzo necesario para su implementaci\' on.
\item   El cliente selecciona qu\' e 
construir, de acuerdo con sus prioridades y las restricciones de tiempo.
\item  El programador construye ese valor de negocio.
     Vuelve al paso 1.\\\\
\end{enumerate}

\begin {center} \textbf{Tabla Comparativa}
\end{center}
\begin{tabular}{||l | c | r||}
\hline
\hline
Caracter\' isticas & Metodolog\' ia RUP & Metodolog\' ia XP \\
\hline
\hline

Obtenci\' on de requisitos& Casos de uso & conversaci\'on con clientes\\
\hline
Carga de Trabajo & Mucho trabajo & Poco Trabajo\\
\hline
Duraci\'on de Proyectos & Larga duraci\' on & Duraci\' on Corta\\
\hline
Detecci\'on de Errores & temprana & A largo plazo\\
\hline
Soporte T\' ecnico & poco continuo & muy continuo\\
\hline
Accesibilidad del cliente al codigo & poca & mucha \\
\hline
Dise\~ no Simple & no & si \\
\hline
Reutilizaci\'on de c\'odigo & si & si \\
\hline
Documentaci\'on & mucha & poca \\
\hline



\end{tabular}






\end{document}